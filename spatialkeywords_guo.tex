\documentclass{sig-alternate}
\usepackage{amsmath}
\usepackage{amssymb}
\usepackage{times}

\usepackage{enumerate}
\usepackage{subfigure}
\usepackage{graphicx}

\usepackage{comment}
\usepackage{cite}
\usepackage{epstopdf}

\pdfpagewidth=8.5in
\pdfpageheight=11in

\makeatletter
\newif\if@restonecol
\makeatother
%
\let\endalgorithm\relax
\usepackage[linesnumbered,ruled,vlined]{algorithm2e}

\newcommand{\reminder}[1]{ [[[ \marginpar{\mbox{$<==$}} #1 ]]] }
\newcommand{\GC}[1]{\reminder{{\bf Gao}: #1}}

\newcommand{\kw}[1]{{\ensuremath {\mathsf{#1}}}\xspace}
\newcommand{\MatchDist}{\mbox{\sf Match}\xspace}
\newcommand{\AnchorMatchDist}{\mbox{\sf AnchorMatch}\xspace}
\newcommand{\PartialEval}{\mbox{\sf PartialEval}\xspace}
\newcommand{\IsMatch}{\mbox{\sf IsMatch}\xspace}
\newcommand{\matchDist}{\mbox{\sf matchDist}\xspace}
\newcommand{\minMatchDist}{\mbox{\sf minMatchDist}\xspace}
\newcommand{\Dist}{\mbox{$\mathsf{Dist}$}\xspace}
\newcommand{\minDist}{\mbox{$\mathsf{minDist}$}\xspace}
\newcommand{\Cost}{\mbox{$\mathsf{Cost}$}}
\newcommand{\Group}{\mbox{$\mathsf{Group}$}}


\newcounter{example}[section]
\renewcommand{\theexample}{\nthesection.\arabic{example}}
\newenvironment{example}{
    \refstepcounter{example}
    {\vspace{1ex} \noindent\bf  Example  \theexample:}}{
    \eop\vspace{1ex}} %\hspace*{\fill}\vspace*{1ex}}


\newcounter{theorem}[section]
\renewcommand{\thetheorem}{\nthesection.\arabic{theorem}}
\newenvironment{theorem}{\begin{em}
    \refstepcounter{theorem}
    {\vspace{1ex} \noindent\bf  Theorem  \thetheorem:}}{
    \end{em}\eop\vspace{1ex}} %\hspace*{\fill}\vspace*{1ex}}
\renewenvironment{proof}{
    {\vspace{1ex} \noindent\bf  Proof:}}{
    \eop \vspace{1ex} }

\newenvironment{lemma}{\begin{em}
    \refstepcounter{theorem}
    {\vspace{1ex} \noindent\bf  Lemma  \thetheorem:}}{
    \end{em}\eop\vspace{1ex}} %\hspace*{\fill}\vspace*{1ex}}


\newcommand{\nthesection}{\arabic{section}}
\newcommand{\eop}{\hspace*{\fill}\mbox{$\Box$}}
\newcommand{\eat}[1]{}
\newcommand{\stitle}[1]{\vspace{0.5ex}\noindent{\bf #1}}

\renewenvironment{proof}{
    {\vspace{1ex} \noindent\bf  Proof:}}{
    \vspace{1ex} }

\newtheorem{defi}{Definition}

\begin{document}
\conferenceinfo{SIGMOD'11,} {June 12--16, 2011, Athens, Greece.}
\CopyrightYear{2011}
\crdata{978-1-4503-0661-4/11/06}
\clubpenalty=10000
\widowpenalty = 10000
%
% --- Author Metadata here ---
%\conferenceinfo{ACM SIGMOD}{'09 Providence, Rhode Island USA}
%\CopyrightYear{2001} % Allows default copyright year (2000) to be over-ridden - IF NEED BE.
%\crdata{0-12345-67-8/90/01}  % Allows default copyright data (0-89791-88-6/97/05) to be over-ridden - IF NEED BE.
% --- End of Author Metadata ---

\title{Collective Spatial Keyword Querying}
%
% You need the command \numberofauthors to handle the "boxing"
% and alignment of the authors under the title, and to add
% a section for authors number 4 through n.
%

%%%%%%%%%%%%%%%
%%%%%%%%%%%%%%%
%%%%%%%%%%%%%%%
%%%%%%%%%%%%%%%


\maketitle


\section{PROCESSING TYPE2 SPATIAL GROUP KEYWORD QUERIES} \label{sec:type2}
\subsection{Circle Limitation to Diameter} \label{secsub:type2:limitation}
Recall the defination of \textsf{TYPE2} problem. The total cost is determined
by two part, and the first part depends on the furthest object while
the second part depends on all pairs of all feasible objects.
For ease of presenting our method, we set parameter $\alpha$ into 0
to disregard the first part cost and discuss the difference
in section ?? when considering this parameter.\par
%
For some given objects, it's possible to find a \textsf{Smallest Enclosing Circle}
which covers all the objects and has the diameter as small as possible. A circle is
helpful for us to make some limitation, because it's obvious that the distance of
any two objects won't exceed the diameter of this smallest enclosing circle $D$.
Therefore, a circle limitation is helpful for us to approach the exact result
of \textsf{TYPE2} problem which is challenging to find the diameter of the keywords-covered
objects. In order to avoid the obscurity of two diameter defination, we call the
diameter of the enclosing circle $D$ and the diameter of the inside objects $L$.\par
%
For known objects it's always possible to find such a circle limitaion,
nevertheless, the problem in \textsf{TYPE2} is the inverse process which aims to
select a subset of all objects with a circle limitaion and covers all the
keywords as well.

\begin{theorem}\label{thm:lambda}
For a set of given points $P$ on the 2-dimensional plane,
the diameter of $P$ is $L$ and the diameter of
$P$'s smallest enclosing circle is $D$. If $|P|>1$ we have
the limition relationship: 
$ \frac{\sqrt{3}}{2} \cdot D\leq L \leq D$

\proof
Some proof here
\end{theorem}

The value of $\frac{\sqrt{3}}{2}$ will be mentional repeatedly,
so we use $\lambda$ to substitute this constant.
According to the above property, it's more clear if we
have the inequality changed into: $L \leq D \leq \frac{L}{\lambda}$.
This means if we find a circle with diameter $D$ and with
feasible objects inside we can make sure that
the result of diameter $L$ must be $[\lambda\cdot D,D]$
without any calculation. Namely, the upperbound and lowerbound
can be utilized to prune in the search space when even
the best lowerbound value can't lead to a better solution.

\begin{lemma}\label{lemma:monotone}
The value of diameter $D$ corresponding to the best solution
is monotonous. That is, if the circle covering the best solution
with a diameter $D$ we can always find a bigger circle with
diameter $D'(D' > D)$ that covers the best solution.

\proof
If set $S$ is the best solution of \textsf{TYPE2} problem,
we could find it's smallest enclosing circle $C$ with a diameter $D$.
The circle $C'$ with diameter $D'$ could fulfill the condition
if put outside the circle $C$.
\end{lemma}

An obvious opinion according to Lemma~\ref{lemma:monotone} is that
we devote to find a circle which both covering all the query keywords 
and with a diameter as small as possible. But according to
Theorem~\ref{thm:lambda} the reverse process is not absolutely
correct if we believe that the circle with the smallest diameter
will cover the best feasible solution.

\subsection{Sweeping Circle} \label{secsub:type2:sweeping}






















\end{document}
