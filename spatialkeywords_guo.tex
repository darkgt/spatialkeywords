\documentclass{sig-alternate}
\usepackage{amsmath}
\usepackage{amssymb}
\usepackage{times}

\usepackage{enumerate}
\usepackage{subfigure}
\usepackage{graphicx}

\usepackage{comment}
\usepackage{cite}
\usepackage{epstopdf}

\pdfpagewidth=8.5in
\pdfpageheight=11in

\makeatletter
\newif\if@restonecol
\makeatother
%
\let\endalgorithm\relax
\usepackage[linesnumbered,ruled,vlined]{algorithm2e}

\newcommand{\reminder}[1]{ [[[ \marginpar{\mbox{$<==$}} #1 ]]] }
\newcommand{\GC}[1]{\reminder{{\bf Gao}: #1}}

\newcommand{\kw}[1]{{\ensuremath {\mathsf{#1}}}\xspace}
\newcommand{\MatchDist}{\mbox{\sf Match}\xspace}
\newcommand{\AnchorMatchDist}{\mbox{\sf AnchorMatch}\xspace}
\newcommand{\PartialEval}{\mbox{\sf PartialEval}\xspace}
\newcommand{\IsMatch}{\mbox{\sf IsMatch}\xspace}
\newcommand{\matchDist}{\mbox{\sf matchDist}\xspace}
\newcommand{\minMatchDist}{\mbox{\sf minMatchDist}\xspace}
\newcommand{\Dist}{\mbox{$\mathsf{Dist}$}\xspace}
\newcommand{\minDist}{\mbox{$\mathsf{minDist}$}\xspace}
\newcommand{\Cost}{\mbox{$\mathsf{Cost}$}}
\newcommand{\Group}{\mbox{$\mathsf{Group}$}}


\newcounter{example}[section]
\renewcommand{\theexample}{\nthesection.\arabic{example}}
\newenvironment{example}{
    \refstepcounter{example}
    {\vspace{1ex} \noindent\bf  Example  \theexample:}}{
    \eop\vspace{1ex}} %\hspace*{\fill}\vspace*{1ex}}


\newcounter{theorem}[section]
\renewcommand{\thetheorem}{\nthesection.\arabic{theorem}}
\newenvironment{theorem}{\begin{em}
    \refstepcounter{theorem}
    {\vspace{1ex} \noindent\bf  Theorem  \thetheorem:}}{
    \end{em}\eop\vspace{1ex}} %\hspace*{\fill}\vspace*{1ex}}
\renewenvironment{proof}{
    {\vspace{1ex} \noindent\bf  Proof:}}{
    \eop \vspace{1ex} }

\newenvironment{lemma}{\begin{em}
    \refstepcounter{theorem}
    {\vspace{1ex} \noindent\bf  Lemma  \thetheorem:}}{
    \end{em}\eop\vspace{1ex}} %\hspace*{\fill}\vspace*{1ex}}


\newcommand{\nthesection}{\arabic{section}}
\newcommand{\eop}{\hspace*{\fill}\mbox{$\Box$}}
\newcommand{\eat}[1]{}
\newcommand{\stitle}[1]{\vspace{0.5ex}\noindent{\bf #1}}

\renewenvironment{proof}{
    {\vspace{1ex} \noindent\bf  Proof:}}{
    \vspace{1ex} }

\newtheorem{defi}{Definition}

\begin{document}
\conferenceinfo{SIGMOD'11,} {June 12--16, 2011, Athens, Greece.}
\CopyrightYear{2011}
\crdata{978-1-4503-0661-4/11/06}
\clubpenalty=10000
\widowpenalty = 10000
%
% --- Author Metadata here ---
%\conferenceinfo{ACM SIGMOD}{'09 Providence, Rhode Island USA}
%\CopyrightYear{2001} % Allows default copyright year (2000) to be over-ridden - IF NEED BE.
%\crdata{0-12345-67-8/90/01}  % Allows default copyright data (0-89791-88-6/97/05) to be over-ridden - IF NEED BE.
% --- End of Author Metadata ---

\title{Collective Spatial Keyword Querying}
%
% You need the command \numberofauthors to handle the "boxing"
% and alignment of the authors under the title, and to add
% a section for authors number 4 through n.
%

%%%%%%%%%%%%%%%
%%%%%%%%%%%%%%%
%%%%%%%%%%%%%%%
%%%%%%%%%%%%%%%


\maketitle


\section{PROCESSING TYPE2 SPATIAL GROUP KEYWORD QUERIES} \label{sec:type2}
\subsection{Circle Limitation to Diameter} \label{secsub:type2:limitation}
For some given objects, it's possible to find a \textsf{Smallest Enclosing Circle}
which covers all the objects and has the diameter as small as possible. A circle is
helpful for us to make some limitation, because it's obvious that the distance of
any two objects won't exceed the diameter of this smallest enclosing circle $D$.
Therefore, a circle limitation is helpful for us to approach the exact result
of \textsf{TYPE2} problem which is challenging to find the diameter of the keywords-covered
objects.

\subsection{Sweeping Circle} \label{secsub:type2:sweeping}






















\end{document}
